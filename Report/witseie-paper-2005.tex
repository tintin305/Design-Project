%%%%%%%%%%%%%%%%%%%%%%%%%%%%%%%%%%%%%%%%%%%%%%%%%%%%%%%%%%%%%%%%%%%%%%%%%%%%%%%
%
% witseiepaper-2005.tex
%
%                       Ken Nixon (12 October 2005)
%
%                       Sample Paper for ELEN417/455 2005
%
%%%%%%%%%%%%%%%%%%%%%%%%%%%%%%%%%%%%%%%%%%%%%%%%%%%%%%%%%%%%%%%%%%%%%%%%%%%%%%%%

\documentclass[10pt,twocolumn]{witseiepaper}

%
% All KJN's macros and goodies (some shameless borrowing from SPL)
\usepackage{KJN}
\usepackage[T1]{fontenc}
\usepackage{amsmath}
\usepackage{pgfgantt}
\usepackage{subcaption}
\usepackage{caption}
\usepackage{siunitx}
%
% PDF Info
%
\ifpdf
\pdfinfo{
/Title (Design Project)
/Author (Tristan Kuisis)
/CreationDate (D:201808300911)
/ModDate (D:200510121530)
/Subject (ELEN417/455 Paper Format, 2005)
/Keywords (Nothing)
}
\fi

%%%%%%%%%%%%%%%%%%%%%%%%%%%%%%%%%%%%%%%%%%%%%%%%%%%%%%%%%%%%%%%%%%%%%%%%%%%%%%%
\begin{document}

\title{Design Project}
\author{Tristan Kuisis
\thanks{School of Electrical \& Information Engineering, University of the
Witwatersrand, Private Bag 3, 2050, Johannesburg, South Africa}
}


%%%%%%%%%%%%%%%%%%%%%%%%%%%%%%%%%%%%%%%%%%%%%%%%%%%%%%%%%%%%%%%%%%%%%%%%%%%%%%%
%
\abstract{%The purpose of this document is to provide overview of the project undertaken from 16th of July to the 24th of August. 
Abstract Here}

\keywords{Keywords}


\maketitle
\thispagestyle{empty}\pagestyle{empty}


%%%%%%%%%%%%%%%%%%%%%%%%%%%%%%%%%%%%%%%%%%%%%%%%%%%%%%%%%%%%%%%%%%%%%%%%%%%%%%%

Do the calculations for an object that is moving with the maximum expected speed (approx 16 km/s) at the altitude that will create the shortest time for it to cross the antennas field of view and then make some assumptions and calculations for this.
%




\section{INTRODUCTION} \label{sec:INTRODUCTION}

% What is an Antenna?
% What is an array?
% What is an antenna array?
% What is space?
% What is space debris? (How did it get there? why is it there? what is it made up of? where is it in space? how is it moving in space? is it bad/good? can we remove it? should we remove it, why? why do we care? who wants to know? fastest moving debris? how often do they come into contact with each other ? what happens when they come into contact with each other? can we protect against it? how is it currently being tracked?  )
% What are orbits?
% What is lower earth orbit? Geostationary orbit?
%  What is the atmosphere? (Range? made up of?)
%  What is the ionosphere? (Range? made up of? characteristics? EM characteristics? How fast can things move through them? )
%  current technology? radar and telescopes? advantages and disadvantages of these technologies?

The nature of an Incoherent Scatter Radar system is such that it directs electromagnetic energy into the earths "surrounding area"(ionosphere?); it highlights irregular characteristics present in this space. The energy that is transmitted is then reflected off of these irregularities and returns back in the direction of the system.
The system has the ability to create a narrow beam which transmits energy, this energy is then sterred (electronically) within the bounds of the system. Steering can be done in azimuth, elevation, and intensity. 
\section{CHOICE OF TECHNOLOGY}

\section{INDIVIDUAL AND ARRAY SIMULATIONS}

\section{COST OPTIMISATION}

\section{CHOICE OF PHYSICAL LOCATION IN SOUTH AFRICA}

\section{WHAT IMPACT WILL THE SYSTEM HAVE ON THE ENVIRONMENT}

\section{SENSITIVITY ANALYSIS}

\section*{ACKNOWLEDGEMENT} \label{sec:ACKNOWLEDGEMENT}


%%%%%%%%%%%%%%%%%%%%%%%%%%%%%%%%%%%%%%%%%%%%%%%%%%%%%%%%%%%%%%%%%%%%%%%%%%%%%%%
%
%\nocite{*}
\bibliographystyle{witseie}
\bibliography{references}


%{\tiny \vfill \hfill \today \hspace{5mm} witseie-paper-2003.\TeX}

\end{document}

" vim: ts=4
" vim: tw=78
" vim: autoindent
" vim: shiftwidth=4
